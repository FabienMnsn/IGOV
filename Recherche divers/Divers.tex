\documentclass{article}

% PACKAGE IMPORT
\usepackage[T1]{fontenc}
\usepackage[utf8]{inputenc}
\usepackage[french]{babel}
\usepackage{sectsty}
\usepackage{lastpage}
\usepackage{fancyhdr}
\usepackage[top=1.25cm, bottom=1.25cm, left=1cm, right=1cm]{geometry}
\usepackage{tabularx,tabulary}
\usepackage{graphicx}
\usepackage{hyperref}


% DOCUMENT START
\begin{document}

% ------------------- PAGES -------------------

\section{Quelques informations de base}
1) Que signifie l'acronyme ICANN?\\
1. Internet Corporation for Assigned Names and Numbers.\\

2) Qui dirige?\\
2.Göran MARBY(Suédois) président de l'ICANN depuis 2016. Il y a eu 5 présidents avant lui\\

3) Ou est le siège?\\
3. Le siège se trouve à Los Angeles en Californie au USA. mais il y a plusieurs autres antennes régionales dans différents pays (Turquie, Belgique, Uruguay).\\

4) Quelle est la mission principale?\\
4. Allouer les noms de domaine et gérer les domaines de premier niveau ainsi que les DNS racines.\\

5) Date de création?\\
5. L'ICANN remplace l'IANA (Internet Assigned Numbers Authority, USA) en 1998 mais reste très proche du gouvernement Américain jusqu'en 2000.\\
 
6) Évolution?\\


7) Qu'est-ce que c'est?\\
7. Une organisation à but non lucratif.\\


8) Infos diverse:
\begin{itemize}
	\item 1. Répartition des salariés dans le monde en 2019:
	\begin{itemize}
		\item 294(75\%) Amérique du Nord
		\item 39(10\%) Europe
		\item 24 (6\%) Moyen Orient et Afrique
		\item 24(6\%) Asie Pacifique
		\item 9 (3\%) Amérique Latine et Caraïbes
		\item TOTAL : 390\\
	\end{itemize}
	\item Évolution du budget de l'ICANN :\\
	\begin{itemize}
		\item en 1998 :
		\item 
	\end{itemize}
\end{itemize}

Brouillon de plan :\\
\begin{itemize}
	\item PAGE DE GARDE + LOGO SSU
	\item TABLE DES MATIÈRES
	\item INTRODUCTION
	\begin{itemize}
		\item Présentation rapide de l'ICANN et de ses fonctions basiques.
		\item Annonce du plan de développement.
	\end{itemize}
	\item PARTIE 1
	\item PARTIE 2
	\item PARTIE N
	\item CONCLUSION \& OUVERTURE\\\\
\end{itemize}



Liens :\\
\begin{itemize}
	\item \url{https://www.icann.org}
	\item \url{https://fr.wikipedia.org/wiki/Internet_Corporation_for_Assigned_Names_and_Numbers}
	\item \url{https://fr.wikipedia.org/wiki/Internet_Assigned_Numbers_Authority}
	\item ...
\end{itemize}
% DOCUMENT END
\end{document}
